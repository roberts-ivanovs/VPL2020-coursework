Darba autors uzskata, ka izstradātā sistēma dara tieši to, ko sākotnēji bija
definējis kā galveno mērķi -- \emph{veikt slimības izplatības modelēšanu}. Kā arī
galvenie apakšuzdevumi ir izpildīti -- ir nodrošināta vizuālā saskarne, iespēja
konfigurēt sistēmu un izpētīt slimības apturošos mehānismsus (karantīnas \emph{pipeline}).

Tomēr, darbu izstrādāt nebija viegli un darba autoram ir vairākas atziņas, kuras
var veikt, gan par pašu projektu, gan C\# kā valodu un .NET ekosistemu un WebAssembly.

\begin{itemize}
    \item Darba autora primārā koda rakstīšanas vide is Visual Studio Code, kas
    nodrošina OmniSharp C\# valodas serveri priekš C\# koda, bet \emph{Razor} koda
    failu apstrāde lika tam regulāri pēkšņi pārtraukt darbību. Diemžēl cita alternatīva
    nebija. Visos pārējos failos tas strādāja ideāli, tomēr laikam tieši šim projektam
    (Blazor/Razor) OmniSharp ir nepieciešama papildus izstrāde.
    \item Darba autors nevarēja atrast labu koda stila linteri, kuru būtu iespējams
    palaist pat no komandrindas, kas pieturētos pie C\# dokumentācijā aprakstītā labā koda stila % TODO reference here
    \item Integrācija ar interneta pārlūkprogrammu no C\# koda ir ļoti slikta.
    Eksistē tiaki viena bibliotēka -- \emph{BlazorExtensions.Canvas} --, kas
    nodrošina pieeju priekš \emph{HTML canvas}, kur WebGL API nestrādāja (varbūt
    tā ir darba autora paša vaina?).
    \item Tā kā \emph{HTML canvas} ir
    JavaScript API, tad katrs funkcijas izsaukums WebAssembly/JavaScript \emph{callback}
    funckiju padošanu ar visiem vajadzīgajiem parametriem -- jo vairāk šādas
    komunikāciajs, jo lielāks leiks overheads rodas.
    \item Canvas ,,2d" nav pietiekami jaudīgs priekš daudzu elementu reālā laika
    zīmēšanas. WebGL būtu bijis labāks risinājums. Canvas ,,2d" API nespēj veikt
    regulāru attēla atjaunošanu bez vizuāliem artefaktiem.
    \item Darba autoram diez gan lielu apjukumu radīja tas, ka \(1 Task\neq 1 Thread\),
    jo liela daļa no projekta bija strukurēta tieši uz šādu pieņēmumu. Tagad zinot
    šo informāciju, maksimālais \emph{Region} skaits noteikti varētu tikt palielināts,
    jo \emph{Region} instancēm ir daudzas bloķējošās darbības (gaidīšana uz
    \emph{Mutex} atbrīvošanos). Tomēr vienmēr jāatstaj pietiekami daudz brīvie
    resursi, lai OS varetu bez problēmām veikt \emph{data polling} un \emph{canvas}
    pārzīmēšanu.
    \item Multithreading aplikācijā, kur ir tikai vizuāla web saskarne, nav vajadzīgs.
    Šādu sistēmu darba autors nekad neieteuktu nevienam likt uz serveriem.
    Darba autors uzskata, ka vislabākais, lai šo aplikāciju tik tiešām arī kāds
    varētu izmantot, būtu nepieciešams pārstrukurēt to kā 1 \emph{Region} aplikāciju
    (bez multithreading) uz \emph{Blazor WebAssembly}. Varbūt, ja nākotnē C\#
    atbalstīs arī jaunākus un eksperimentālākus standartus, tad integrācija ar
    \emph{WebWorkers} būtu iespējama, lai veiktu ssitēmas paralelizāciju ar vairākiem
    threadiem -- bet šobrīd tas vēl nav iespējams.
    \item Darba autoram pipeline konstrukcija likās viselegantākais risinajums
    šādai modulārai problēmai, it sevišķi izmantojot \emph{LINQ}, priekš īsāka un lasāmāka
    koda. Ja otreiz būtu jātaisa šāda sistēma, tad šo koda
    dizaina elementu noteikti arī nestu uz jauno sistēmu.
    \item Mutex konstrukcija ir obligāti nepieciešama, lai taisītu aplikāciju
    bez data-razes, tomēr C\# valoda ir konstruēta bez papildus kompilācijas
    drošības nosacījumiem un Mutex izmantošana ir ,,programmētāja labākajās
    interesēs bet ne obligāta", kas var rast daudzas dažādas kļūdas, ja kāds mainīgais
    nav pareizi aizsargāts.
\end{itemize}

Darba autors uzskata, ka mūsdienās eksitē labāki alternatīvi šāda projekta
izstrādei teju vai visos iespējamajos scenārijos. Tā arī būs visticamāk šī
projekta nākotne -- mēgināt to implementēt ar citām valodām un rīkiem, lai veiktu
salīdzinājumus.
