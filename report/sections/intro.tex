
% Problēma ko risina
\subsection{Problēma}

Viens no 2020 gada lielkājiem izaicinājumiem ir bijusi globālā \emph{COVID-19} pandēmija.
Šī slimība ir ļoti infekcioza, bet ne ļoti nāvējoša (vismaz ne jauniem cilvēkiem), tomēr
tā spēj ļoti ātri izsūkt visus spēkus no valsts medicīniskās sistēmas. Lai valdības spētu
pieņemt lēmumus par ierobežojumu ieviešanu, ļoti vērtīgi ir veikt dažādas simulācijas,
lai saprastu, kādā veidā slimība varētu progresēt un izplatīties sabiedrībā.

Šāda problēma noved pie projekta galvenā mērķa - \emph{veikt slimības izplatības modelēšanu}.

Mērķis tiks uzskatīts par sasniegtu, kad būs izpildīti sekojošie apakšuzdevumi:

\begin{enumerate}
    \item Vizuāla reprezentācija slimības procesa gaitai
    \item Iespēja pielietot slimības apturēšanas mehānismus
    \item Priekšnosacījumu konfigurēšana, lai ietekmētu simulācijas iznākumus
\end{enumerate}

% Tehniskie (personīgie) izaicinājumi
\subsection{Tehniskie izaicinājumi}

Ņemot vērā, ka kursadarba nosacījumos ir teikts, ka ir jaizmanto 3 dažādas kursa
laikā apskatītas tehnoloģijas, tad darba autors ir izvēlējies sekojošas tēmas:

\begin{itemize}
    \item \emph{Blazor} - moderna tehnoloģija, kas darba autoram likās interesanta.
        Alternatīvs - \emph{WPF} - nestrādā uz Linux, tāpēc tas nemaz netika apsvērts kā izmantojams variants
    \item Multithreading - darba autoram iepriekš nav bijusi pieredze ar vairāku-kodolu
        aplikāciju programmēšanu un uzskatīja, ka šis kursadarbs būs lielisks veids kā papildus apgūt zināšanas šajā jomā
    \item \emph{LINQ} - darba autoram ir iepriekšēja pieredze citās programmēšanas
        valodās ar dažādām funkcionalo valodu īpašībām, kā arī, ņemot vērā, ka šis
        kursadarbs būs ļoti \emph{uz datiem vērsts}, tad \emph{LINQ} ir loģiska izvēle šo datu apstrādei.
\end{itemize}
