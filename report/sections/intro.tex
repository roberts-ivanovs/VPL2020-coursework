
% Problēma ko risina
\subsection{Problēma}

Viens no 2020. gada lielākajiem izaicinājumiem ir bijusi globālā \emph{COVID-19} pandēmija.
Šī slimība ir ļoti infekcioza\cite{covid:contagious}, bet ne ļoti nāvējoša (vismaz ne jauniem cilvēkiem), tomēr
tā īsā laika sprīdī var pārslogot valsts medicīniskās sistēmas\cite{covid:medical-system-strain}. Lai valdības spētu
pieņemt lēmumus par ierobežojumu ieviešanu, ļoti vērtīgi ir veikt dažādas simulācijas,
lai saprastu, kādā veidā slimība varētu progresēt un izplatīties sabiedrībā, un kas
varētu mainīties, ja tiek ieviesti ierobežojumi.

Šāda problēma noved pie projekta galvenā mērķa - \textbf{veikt slimības \(x\) izplatības modelēšanu}.

Mērķis tiks uzskatīts par sasniegtu, kad būs izpildīti sekojošie uzdevumi:

\begin{enumerate}
    \item Vizuāla reprezentācija slimības procesa gaitai.
    \item Iespēja pielietot slimības apturēšanas mehānismus.
    \item Priekšnosacījumu konfigurēšana, lai ietekmētu simulācijas iznākumus.
\end{enumerate}

% Tehniskie (personīgie) izaicinājumi
\subsection{Tehnoloģijas}

Ņemot vērā, ka kursa darba nosacījumos ir teikts, ka ir jāizmanto 3 dažādas kursa
laikā apskatītas tehnoloģijas, tad darba autors ir izvēlējies sekojošās tehnoloģijas:

\textbf{Blazor Server}\cite{blazor:info} - moderna \emph{WEB} bāzēta tehnoloģija, kas darba autoram likās interesanta.
Darba autoram ir pieredze \emph{WEB} sistēmu izstrādē un liela nepatika
pret \emph{JavaScript} programmēšanas valodu. Darba autors vēlējās sīkāk izpētīt
alternatīvus \emph{WEB UI} veidošanā. Otrs apsvērtais
alternatīvs \emph{UI} veidošanai, kas tika apskatīts studiju kursā - \emph{WPF}\cite{wpf:info} - nestrādā uz
Linux, tāpēc to darba autors nevēlējās izmantot.

\textbf{Multithreading} - darba autoram iepriekš nav bijusi pieredze ar vairāku procesu
aplikāciju/sistēmu programmēšanu un uzskatīja, ka šis kursa darbs būs lielisks
veids kā apgūt zināšanas šajā jomā, iepazīstinātu sevi ar dažādiem konceptiem, kā
informācijas dalīšanu starp \emph{threadiem} un sistēmas paralēlizēšanu.

\textbf{LINQ}\cite{linq:info} - darba autoram ir iepriekšēja pieredze citās programmēšanas
valodās ar dažādām funkcionālo valodu īpašībām, kā arī, ņemot vērā, ka šis
kursa darbs ir ļoti ,,uz datiem vērsts", tad \emph{LINQ} ir loģiska izvēle šo datu
ērtai apstrādei un filtrēšanai \emph{C\#} programmēšanas valodā.
