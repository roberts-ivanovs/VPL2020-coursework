
% Problēma ko risina
\subsection{Problēma}

% TODO Add source of reference šitam paragrāfam
Viens no 2020 gada lielkājiem izaicinājumiem ir bijusi globālā \emph{COVID-19} pandēmija.
Šī slimība ir ļoti infekcioza, bet ne ļoti nāvējoša (vismaz ne jauniem cilvēkiem), tomēr
tā spēj ļoti ātri izsūkt visus spēkus no valsts medicīniskās sistēmas. Lai valdības spētu
pieņemt lēmumus par ierobežojumu ieviešanu, ļoti vērtīgi ir veikt dažādas simulācijas,
lai saprastu, kādā veidā slimība varētu progresēt un izplatīties sabiedrībā.

Šāda problēma noved pie projekta galvenā mērķa - \emph{veikt slimības izplatības modelēšanu}.

Mērķis tiks uzskatīts par sasniegtu, kad būs izpildīti sekojošie apakšuzdevumi:

\begin{enumerate}
    \item Vizuāla reprezentācija slimības procesa gaitai
    \item Iespēja pielietot slimības apturēšanas mehānismus
    \item Priekšnosacījumu konfigurēšana, lai ietekmētu simulācijas iznākumus
\end{enumerate}

% Tehniskie (personīgie) izaicinājumi
\subsection{Tehnoloģijas}

Ņemot vērā, ka kursadarba nosacījumos ir teikts, ka ir jaizmanto 3 dažādas kursa
laikā apskatītas tehnoloģijas, tad darba autors ir izvēlējies sekojošas tēmas:

\textbf{Blazor Server}\cite{blazor:info} - moderna WEB bāzēta tehnoloģija, kas darba autoram likās interesanta.
Interesei radīja, tas, ka darba autoram ir pieredze WEB sistēmu izstrādē un liela nepatika
pret JavaScript programmēšanas valodu, lai veidotu lietotāju saskarnes, uzskatot,
ka Blazor varētu būt labs alternatīvs nākotnes projektiem. Otrs apsvērtais
alternatīvs UI veidošanai, kas tika apskatīts studiju kursā - WPF\cite{wpf:info} - nestrādā uz
Linux, tāpēc to darba autors nevēlējās izmantot.

\textbf{Multithreading} - darba autoram iepriekš nav bijusi pieredze ar vairāku procesu
aplikāciju/sistēmu programmēšanu un uzskatīja, ka šis kursadarbs būs lielisks
veids kā apgūt zināšanas šajā jomā, iepazīstinātu sevi ar dažādiem konceptiem, kā
informācijas padošānu threadiem, objektu kopīgošanu un sistēmas paralelizēšanu.

\textbf{LINQ}\cite{linq:info} - darba autoram ir iepriekšēja pieredze citās programmēšanas
valodās ar dažādām funkcionalo valodu īpašībām, kā arī, ņemot vērā, ka šis
kursadarbs ir ļoti \emph{uz datiem vērsts}, tad LINQ ir loģiska izvēle šo datu
ērtai apstrādei un filtrēšanai ar C\# kodu.
