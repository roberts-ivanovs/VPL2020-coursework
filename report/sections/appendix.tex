\section{Pipeline interfeiss}
\label{app:IPipeline}
{
\setstretch{1}
\begin{minted}{CSharp}
    public interface Pipeline
    {
        void updateTimeScale(float timeScale);
        PipelineReturnData pushThrough(
            List<EntityOnMap<SickEntity>> currentSick,
            List<EntityOnMap<HealthyEntity>> currentHealthy,
            ulong timeDeltaMs
        );
    }
\end{minted}
}

\section{Pipeline datu izfiltrēšana ar LINQ}
\label{app:pipelien-linq}
{
\setstretch{1}
\begin{minted}{CSharp}
    /* Izrēķina laika starpību kopš iepriekšējās iterāciajs */
    var current = sw.ElapsedMilliseconds;
    var timeDeltaMS = (ulong)(current - previous);
    /* Iterē cauri visiem pipeliniem */
    var pipelineResult = pipelines.Aggregate(new PipelineReturnData
    {
        // Inicializē ar sākotnējām Region vērtībām
        newHealthy = populationHealthy,
        newSick = populationSick
    },
    (aggregate, pipeline) => pipeline.pushThrough(
        aggregate.newSick, aggregate.newHealthy, timeDeltaMS
        )
    );

    // Saglabā izmainītās vērtības
    populationSick = pipelineResult.newSick;
    populationHealthy = pipelineResult.newHealthy;
\end{minted}
}
