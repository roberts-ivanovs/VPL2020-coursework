\subsection{Projekta struktūra}

Pirms sākot faktisko programmēšanu, darba autors veltīja laiku projekta
struktūras izplānošanai, lai veiksmīģiv arētu projetu realizet. Viens no šādiem
galvenajiem mērķiem bija atdalīt vizuālo lietotāju saskarni no pašas simulācijas.
C\# piedāvā diez gan labu risinājumu šādas projetkas struktūras attīstīšanai - izmantojot
\emph{namespaces} \cite{csharp:namespaces}. Tādējādi projekts tika strukturēts
divās galvenajās pakotnēs:

\begin{itemize}
    \item \emph{DiseaseCore} - atbildīgs par simulācijas veikšanu un programmiskā
        interfeisa piedāvāšanu, simulācijas veikšanai
    \item \emph{frontendserver} - atbildīgs par lietotāju saskarni
\end{itemize}

Šādu pieeju darba autors sākotnēji izvēlējās, jo nebija vēl pilnībā pārliecināts
par to, vai ar Blazor būs iespējams realizēt vēlamo projektu, tādēļ šadi atdalot
lietotāju saskarni no simulācijas koda, ar nosacījumu, ka simulācijas kods tiek
uzrakstīts pirms UI koda, varēs ērtāk \emph{pārlekt kuģus}, ja Blazor nebūs
iespējams izmatnot darbam un vajadžēs izmantto WPF. Lai gan darba izstrāde
izmantojot Blazor bija veiksmīga, tomēr šāda koda atdalīšana padarīja to ērtāk
menedžējamu, kad biaj nepieciešams veikt izmaiņas kodā, jo, paturot DiseaseCore
publisko interfeisu nemainīgu, bija ērti veikt izmaiņas pāšā simulācijas loģikā
pat pēc UI izstrādes, neradot papildus liekas koda kompilēšanas kļūdas.

\subsubsection*{Simulācijas koda struktūra}
% TODO Finish this


\subsubsection*{Lietotāja saskarnes koda struktūra}
% TODO Finish this
