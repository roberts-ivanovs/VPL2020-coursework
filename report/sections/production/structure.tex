Pirms uzsākot faktisko programmēšanu, darba autors veltīja laiku projekta
struktūras izplānošanai, lai veiksmīgi varētu projektu realizēt. Viens no šādiem
galvenajiem mērķiem bija atdalīt vizuālo lietotāju saskarni no pašas simulācijas.
\emph{C\#} piedāvā diez gan labu risinājumu šādas projekta struktūras problēmas
attīstīšanai - izmantojot
\emph{namespaces}\cite{csharp:namespaces}. Tādējādi projekts tika strukturēts
divās galvenajās pakotnēs:

\begin{itemize}
    \item \emph{DiseaseCore} - atbildīgs par simulācijas veikšanu un programmatiskā
        interfeisa piedāvāšanu, kas piedāvātu informāciju par simulācijas statusu.
    \item \emph{frontendserver} - atbildīgs par lietotāju saskarni un simulācijas
        \emph{DiseaseCore} inicializāciju.
\end{itemize}

Šādu pieeju darba autors sākotnēji izvēlējās, jo nebija vēl pilnībā pārliecināts
par to, vai ar \emph{Blazor} būs iespējams realizēt vēlamo projektu. Šādi atdalot
lietotāju saskarni no simulācijas koda, ar nosacījumu, ka simulācijas kods tiek
uzrakstīts pirms \emph{UI} koda -- ja ar \emph{Blazor} rastos neparedzētas problēmas
(jauna, pilnībā neizpētīta tehnoloģija no darba autora skatpunkta ), tad varēs ērtāk
nomainīt vizuālās saskarnēs ietvarus \(Blazor \rightarrow WPF\).

Lai gan darba izstrāde, izmantojot \emph{Blazor} bija veiksmīga, tomēr šāda koda loģiskā
atdalīšana padarīja to ērtāk
pārvaldāmu, kad bija nepieciešams veikt izmaiņas kodā, jo, paturot \emph{DiseaseCore}
publisko interfeisu nemainīgu, bija ērti veikt izmaiņas pašā simulācijas loģikā
pat pēc \emph{UI} izstrādes, neradot papildus liekas koda kompilēšanas kļūdas.

\subsubsection*{Simulācijas koda vispārīga būtība}

Simulācijas kods nodrošina sekojošās pamatdarbības:

\begin{itemize}
    \item Veselu entītiju dzīvesciklu
    \item Slimu entītiju dzīvesciklu
    \item Entītiju savstarpēju mijiedarbību
\end{itemize}

Ideja ir reprezentēt katru entītiju kā ierobežotu punktu plaknē, aizliedzot entītijām šķērsot
plaknes robežas. Katrām entītijām ir pieeja saviem iekšējajiem parametriem - virzienu
uz kuru dodas (reprezentē ar vektoru), vecumu, veselības stāvokli. Simulācijas
kods atļauj ērtā veidā definēt darbības ar entītijām - infekcijas izplatīšanos,
atveseļošanos, noteiktu \emph{mērķu} (kodā saukti par \emph{attractors}) definēšanu
plaknē -- entītijām būs savas rutīnas, ceļojot starp šiem mērķiem; tas kalpo kā
vienkāršs veids, lai simulētu veikalus/slimnīcas/skolas u.c. publiskas telpas,
kur entītijas dodas tikties. Tiek arī nodrošināta iespēja slimās entītijas
ielikt karantīnā, līdz tās atveseļojas. Viena no darba autora iecienītākajām
funkcijām ir \emph{zombie mode}, kas liek slimajiem entītijām dzīties
pakaļ veselajām, līdz visa populācija saslimst (lai gan šis nebija no izvirzītajiem
mērķiem, to ir interesanti skatīties vizuālajā saskarnē).

\subsubsection*{Lietotāja saskarnes koda vispārīgā būtība}

Lietotāja saskarnes galvenā ideja ir parādīt entītijas kaut kādā plaknē, tos
iekrāsot uzskatāmā veidā, un nodrošināt dažādu simulācijas funkciju izslēgšanu
un ieslēgšanu priekš simulācijas (piemēram \emph{karantīna}, \emph{zombie mode},
\emph{slimības pārnēsāšanās}, utt.). Kā arī tiek nodrošināta papildus meta-datu
parādīšana lietotājam par to kā strādā esošā sistēma -- statistikas sadaļa.

Vizuālā saskarne nav ļoti sarežģīta, jo tās galvenais mērķis ir parādīt simulācijas
informāciju.
