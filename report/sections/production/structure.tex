Pirms sākot faktisko programmēšanu, darba autors veltīja laiku projekta
struktūras izplānošanai, lai veiksmīģiv arētu projetu realizet. Viens no šādiem
galvenajiem mērķiem bija atdalīt vizuālo lietotāju saskarni no pašas simulācijas.
C\# piedāvā diez gan labu risinājumu šādas projetkas struktūras attīstīšanai - izmantojot
\emph{namespaces} \cite{csharp:namespaces}. Tādējādi projekts tika strukturēts
divās galvenajās pakotnēs:

\begin{itemize}
    \item \emph{DiseaseCore} - atbildīgs par simulācijas veikšanu un programmiskā
        interfeisa piedāvāšanu, simulācijas veikšanai
    \item \emph{frontendserver} - atbildīgs par lietotāju saskarni
\end{itemize}

Šādu pieeju darba autors sākotnēji izvēlējās, jo nebija vēl pilnībā pārliecināts
par to, vai ar Blazor būs iespējams realizēt vēlamo projektu, tādēļ šadi atdalot
lietotāju saskarni no simulācijas koda, ar nosacījumu, ka simulācijas kods tiek
uzrakstīts pirms UI koda, varēs ērtāk \emph{pārlekt kuģus}, ja Blazor nebūs
iespējams izmatnot darbam un vajadžēs izmantto WPF. Lai gan darba izstrāde
izmantojot Blazor bija veiksmīga, tomēr šāda koda atdalīšana padarīja to ērtāk
menedžējamu, kad biaj nepieciešams veikt izmaiņas kodā, jo, paturot DiseaseCore
publisko interfeisu nemainīgu, bija ērti veikt izmaiņas pāšā simulācijas loģikā
pat pēc UI izstrādes, neradot papildus liekas koda kompilēšanas kļūdas.

\subsubsection*{Simulācijas koda vispārīga būtība}

Simulācijas kods nodrošina sekojošās pamatdarbības:

\begin{itemize}
    \item Veselu entītiju dzīvesciklu
    \item Slimu entītiju dzīvesciklu
    \item Entītiju savstarpēju mijiedarbību
\end{itemize}

Ideja ir reprezentēt katru entītiju kā punktu plaknē (izstrādes
laikā tika pieņemts lēmums plakni tomēr ierobežot, aizliedzot entītijiem šķērsot
noteiktas robežas). Katrs entītijs zina noteiktas vērtības par sevi - virzienu
uz kuru dodas (reprezentē ar vektoru), vecumu, veselības stāvokli. Simulācijas
kods atļauj ērtā veidā definēt darbības ar entītijiem - infekcijas izplatīšanos,
atveseļošanos, noteiktu \emph{mērķu} definēšanu plaknē pa kuriem entītiji ceļos
(vienkāršs veids lai simulētu veikalus/slimnīcas/skolas u.cc publiskas telpas
kur entītiji dodas tikties). Tiek arī nodrošināta iespēja slimos entītijus
ielikt karantīnā, līdz tie atveseļojas. Viena no darba autora iecienītākajām
funkcijām ir ,,Zombie mode", kas liek slimajiem entītijiem dzīties
pakaļ veselajiem, līdz visa populācija saslimst (lai gan šis nebija no izvirzītajiem mērķiem).

\subsubsection*{Lietotāja saskarnes koda vispārīgā būtība}

Lietotāja saskarnes galvenā ideja ir parādīt entītijus kaut kādā plaknē, tos
iekrāsot uzskatāmā veidā, un nodrošināt dažādo funckiju izslēgšanu un ieslēgšanu
priekš simulācijas (piemēram \emph{karantīna}, \emph{zombie mode},
\emph{slimības pārnēsāšanās}, utt.). Kā arī tiek nodrošināta papildus meta-datu
parādīšana lietotājam par to kā strādā esošā sistēma.

Vizualā saskarne nav ļoti sarežģīta, jo tās galvenais mērķis ir parādīt simulāciajs
informāciju.
