
% 2 sadaļas, footeris, main content

Izstrādāto galaprojektu ir iespējams uzstartēt ar \emph{shell} skrptu projekta
pamatdirektorijā \texttt{./start.sh}.
Projektu uzstartējot, apmeklējot ar interneta pārlūkprogrammu saiti
\url{https://127.0.0.1:5001/}, atvērsies lapa, kāda attēlota \ref{img:whole-ui} attēlā.

\begin{figure}[H]
	\centering
	\includegraphics[scale=0.4]{images/ui-whole-page.png}
	\caption{Lietotāju saskarnes sākuma skats}
	\label{img:whole-ui}
\end{figure}

Lietotāju saskarne sastāv no divām lapām - pašas simulācijas (\ref{img:whole-ui}
att.) un ,,About" sadaļas (nav attēlots atskaitē).

% Settings panelis
\subsection{Iestatījumu panelis}

,,Settings" panelis ir galvenā vieta saskarnē, kur ir iespējams veikt sekojošās darbības:

\begin{itemize}
    \item \emph{Pipeline} ieslēgšanu/izslēgšanu,
    \item populācijas skaita maiņu,
    \item slimo entītiju skaita maiņu,
    \item datu pieprasīšanas intervāla (\emph{polling rate}) maiņu,
    \item iespēja piespiedus kārtā izmantot tikai vienu \emph{Region} instanci
    \item Uzsākt/apstādināt simulāciju
\end{itemize}

% Canvas
\subsection{Kartes komponente}

\emph{Canvas} elements attēlo simulācijā esošos entītijus lietotājam. Ņemot vērā, ka
simulācijas plaknes izmēri nesakrīt ar pikseļu dimensijām lietotāju saskarnes
\emph{canvas} elementa izmēriem, tad katrs punkts arī tiek normalizēts jaunajās
(lietotājam redzamajās) dimensijās. Zināmas problēmas gan rodas gadījumos, kur
ir liels entītiju skaits -- \emph{canvas 2d context} nespēj pietiekami ātri
uzzīmēt lielo elementu skaitu, rodas vizuāli artefakti. \emph{Canvas} elementus
ar dažādiem simulāciajs stāvokļiem dažādās stadijās var apskatīt \ref{img:canvas-show-off} attēlā.
Slimo entītiju krāsa mainās atkarībā no viņu veselības stāvokļa -- jo gaišāks, jo sliktāka veselība.

\begin{figure}[H]%
    \centering
    \subfloat
        [\centering Karantīna tiek ievērota (\emph{Healthy attractors} ieslēgts)]
        {{\includegraphics[width=.3\linewidth]{images/quarantine.png} }}%
    \qquad
    \subfloat
        [\centering Slimība izplatās, ja karantīna netiek ievērota]
        {{\includegraphics[width=.3\linewidth]{images/sickness-spreads.png} }}%
    \subfloat
        [\centering Atraktori ir izslēgti, ieslēgts \emph{Zombie mode}]
        {{\includegraphics[width=.3\linewidth]{images/zombiemode-1.png} }}%
    \qquad
    \subfloat
    [\centering Ilgtermiņā turot ieslēgtu \emph{Zombie mode}, var redzet \emph{Region} sadalījumu]
        {{\includegraphics[width=.5\linewidth]{images/zombiemode-2.png} }}%
    \caption{Dažādu simulāciajs stāvokļu viuzalizācija}%
    \label{img:canvas-show-off}%
\end{figure}

% Statistics
\subsection{Statistikas komponente}

Tā kā \emph{canvas} vizualizācija der tikai kā uzskates materiāls, \emph{Statistics}
sadļa atļauj apskatīt to, kas notiek sistēmā ,,caur skaitļiem" (skat. \ref{img:statistics} att.).
Šī sadaļa ir sadalīta 3 kolonnās -- entītiju vispārīgā statistika, katra reģiona
noslodze (tiek mērīta ar iterācijām sekundē, kur 1 iterācija ir visu reģiona entītiju
izlaišana caur \emph{pipeline}, \emph{inbound} bufera pārbaude, entītiju ievietošana
\emph{out of bounds} buferī), bet 3. kolonna ir vidējāis aritmētiskais visu reģionu
noslodzei kopš simulācijas sākuma. Šie skaitļi atļauj ērti analizēt sistēmas
izmaiņas pie dažādiem iestatījumiem.

\begin{figure}[H]
	\centering
	\includegraphics[scale=0.5]{images/statistics.png}
	\caption{Statistikas sadaļa}
	\label{img:statistics}
\end{figure}

\subsection{Vairāku-kodolu stresa tests}
% Max entities, no deaths, no nothing 1REGION vs 5 REGIONS
Izmantojot šo informāciju, kas apskatāma \emph{Statistics} komponentē, var veikt
nelielus testus esošajai programmai. Sākotnējais entītiju skaits tika uzlikts uz
maksimālo, lai pēc iespējas vairāk noslogotu procesoru, tad tika palaista programma
uz \~ 20 sekundēm, tikai ievākta informācija no sistemas. Šāda procedūra tika
atkārtota, izmantojot 1 \emph{Region} un 5 \emph{Region} instances. Rezultātus
ir iespējams apskatīt \ref{img:test-single-vs-mutlicore} attēlā.


\begin{figure}[H]%
    \centering
    \subfloat
        [\centering Statistika ar 1 \emph{Region} instanci]
        {{\includegraphics[width=.7\linewidth]{images/stress-statistics-single-sm.png} }}%
    \qquad
    \subfloat
        [\centering Statistika ar 5 \emph{Region} instancēm]
        {{\includegraphics[width=.7\linewidth]{images/stress-stastiscs-multiple.png} }}%
    \caption{Sistēmas stresa tests}
    \label{img:test-single-vs-mutlicore}%
\end{figure}

Uz šo informāciju, ko attēlo šī komponente ir jāvērtē tikai kā ,,ieskats". Katru
reizi ģenerējot jaunu simulāciju, entītiji tiek uzģenerētei citās vietās, ar
dažādiem parametriem, atraktori tiek definēti dažādos daudzumos un dažādās vietās.
Protams, to visu ir iespējams statiski iekodēt kodā, lai tas nemainās, tomēr,
izstradājot sistēmu, mērķis nebija veikt padziļinātu analīzi par paralelizācijas
uzlabojumiem veiktspējā (ironiski, bet tiešī šī apakšadaļa to arī dara).
